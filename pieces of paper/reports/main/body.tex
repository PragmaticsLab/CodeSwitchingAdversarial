\subsection{Обучение моделей на датасете MultiAtis++}
В своей работе мы обучаем языковые модели решать задачу задачи одновременного детектирования намерений пользователя и заполнения слотов для диалоговых помощников, направленных на выполнение конкретной задачи.
Эта задача заключается в классификации предложений и всех слов в предложении.

\subsubsection{Датасет}
В качестве датасета в своей работе мы выбрали датасет MultiAtis++~\cite{Xu2020EndtoEndSA}.
В этом датасете представлены семь языков из трёх языковых семей —
Индо-Европейская (английский, немецкий, французский, испанский, португальский), Японо-рюкюская (японский) и Сино-тибетская (китайский).
Датасет является параллельным корпусом для задачи классификации интентов и разметки слотов - в 2020 году он был переведён с английского языка на остальные шесть.
В обучающей выборке содержится 4978 предложений для каждого языка, в тестовой 893 предложения для каждого языка.

\begin{table}[H]
    \resizebox{\textwidth}{!}{
        \begin{tabular}{|>{\bfseries}c|ccccccc|}
            \hline
            Intent & \multicolumn{7}{c|}{ atis\_flight } \\ \hline
            Utterance en   & show  & me  & flights & from & montreal             & to   & orlando            \\ \hline
            Slot labels en & O     & O   & O       & O    & B-fromloc.city\_name & O    & B-toloc.city\_name \\ \hline
            Utterance de   & Zeige & mir & Flüge   & von  & Montreal             & nach & Orlando            \\ \hline
            Slot labels de & O     & O   & O       & O    & B-fromloc.city\_name & O    & B-toloc.city\_name \\ \hline
        \end{tabular}
    }\caption{Пример объекта из датасета MultiAtis++. На примере представлен объект на английском и немецком языке.}\label{tab:table}
\end{table}
\parКаждый объект в датасете состоит из предложения, меток слов в BIO формате и интента (Таблица~\eqref{tab:table}).
Перед началом работы с датасетом мы произвели предварительную очистку —
убрали из обучающей и тестовой выборок объекты, для которых на любом из семи языков количество слов и количество слотов не совпадали.
Таким образом, в обучающей выборке осталось 4884 объекта для каждого языка, в тестовой выборке 755 объектов для каждого языка.
Для составления списка используемых слотов и интентов использовалась обучающая выборка на английском языке.
Мы использовали 121 различную метку слотов и 23 различных метки интентов.
Список id используемых объектов, а также списки используемых слотов и интентов можно найти в приложении.

\subsubsection{Архитектура модели}
В своей работе мы решаем задачу одновременной классификации интентов и разметки слотов в предложении с помощью одной модели.
Модель имеет два выхода, первый предсказывает интенты, второй предсказывает метки слов.
В качестве рассматриваемых архитектур были выбраны модели m-BERT~\cite{devlin-etal-2019-bert} и XLM-RoBERTa~\cite{Conneau2020UnsupervisedCR}.
Обе эти модели являются одними из самых сильных мультиязычных моделей на текущий момент.
Каждая из них предобучена на более чем ста языках.
\parОбозначим количество блоков Трансформера за L, размер скрытых представлений за H и количество голов с внутренним вниманием за A\@.
Тогда в используемой нами модели m-BERT L = 12, H = 768, A = 12, а суммарное количество параметров 110 миллионов.
В используемой нами модели XLM-RoBERTa L = 12, H = 768, A = 12, а суммарное количество параметров 270 миллионов.

\subsubsection{Обучение}
В своей работе мы будем сравнивать модели, обученные на всей обучающей выборке и только на части обучающей выборки на английском языке.
Таким образом мы сможем проверить насколько устойчивы к нашим атакам модели с разными вариантами обучения.
\parКаждая из моделей обучалась с одинаковыми гиперпараметрами - 10 эпох на обучающей выборке с длиной шага обучения $10^{-5}$ и размером батча в 64 объекта.
В качестве функции ошибки использовалась кросс-энтропия:
\begin{equation}
    L = -\dfrac{1}{n}\sum\limits_{i = 1}^{n}\left[ y\log\left(\hat{y}\right) \right]\label{eq:equation4}
\end{equation}
\parВ своей работе мы будем использовать следующие метрики качества:
\begin{itemize}
    \item Доля предложений, в которых правильно классифицирован интент:
    \begin{equation}
        \textbf{Intent accuracy} = \# \text{sentences} \left[ \left(I_{pred} = I_{true} \right) \right]\label{eq:equation}
    \end{equation}
    \item F1 мера для меток слотов (используется микро-усреднение по всем классам):
    \begin{equation}
        \textbf{Slots F1 score} = 2 \cdot \dfrac{\text{precision} \cdot \text{recall}}{\text{precision} + \text{recall}}\label{eq:equation2}
    \end{equation}
    \item Доля предложений, в которых правильно классифицирован интент и верно классифицированы все слоты:
    \begin{equation}
        \textbf{Semantic accuracy} = \# \text{sentences} \left[ \left(I_{pred} = I_{true} \right) \land \left(S_{pred} = S_{true} \right)\right]\label{eq:equation3}
    \end{equation}
\end{itemize}

\subsection{Адверсариальные атаки}
В своей работе мы предлагаем два варианта gray-box адверсариальных атак — во время выполнения атаки мы имеем доступ к ошибке модели.
Мы стремимся создать атаку такого рода, чтобы результирующая адверсариальная пертурбация предложения была как можно ближе к реалистичным предложениям со смешением кодов.
Для этого мы заменяем часть токенов в предложении на их эквиваленты из других языков.
Оценка качества на таких адверсариальных атаках может выступать в роли оценки снизу на качество соответствующих моделей в аналогичных задачах при наличии реального смешения кодов во входных данных.
\parТак как большинство людей, которые могут использовать смешение кодов в своей речи, билингвы, то в основном смешение кодов происходит между парой языков~\cite{bilinguals}.
Таким образом, в своей работе мы предлагаем анализировать атаки состоящие во встраивании одного языка в другой.

\subsubsection{Общий вид атаки}
Общий принцип атаки одинаковый для обоих предлагаемых вариантов.
Разница между методами заключается в способе генерации кандидатов на замену токену на $i-$ой позиции.
В своей работе мы предлагаем следующий вид атаки — пусть мы имеем целевую модель, пару пример-метка и встраиваемый язык (Алгоритм~\eqref{alg:algorithm}).
Тогда мы перебираем токены в предложении в случайном порядке и стремимся заменить токен на его эквивалент из встраиваемого языка.
Если это приведёт к увеличению ошибки модели, то мы заменяем токен на предложенного кандидата.

\begin{algorithm}
    \caption{Общая схема адверсариальной атаки}
    \begin{algorithmic}
        \Require{Пара пример-метка x, y; целевая модель $\mathcal{M}$; встраиваемый язык $\mathbb{L}$}
        \Ensure{Адверсариальный пример x'} \\
        $\mathcal{L}_{x}$ = GetLoss($\mathcal{M}$, x, y)
        \For{i in permutation(len(x))}
            \\
            \ind Candidates = GetCandidates($\mathcal{M}$, x, y, token\_id = i) \\
            \ind Losses = GetLoss($\mathcal{M}$, Candidates)
            \ind\If{Candidates and max(Losses) > $\mathcal{L}_{x}$}
                    \\
                    \ind\ind$\mathcal{L}_{x}$ = max(Losses) \\
                    \ind\ind x, y = Candidates[argmax(Losses)]
            \EndIf
        \EndFor \\
        \ind\Return x
    \end{algorithmic}\label{alg:algorithm}
\end{algorithm}

\subsubsection{Word level атака}
Первый предлагаемый нами вариант атаки заключается в генерации эквивалентов из других языков с помощью перевода токенов на соответствующие языки (Алгоритм~\eqref{alg:algorithm1}).
Атакуя таким образом, мы строим грубую оценку снизу, так как при атаке мы не учитываем контекста предложений и не учитываем многозначность слов.
Этот вариант схож с атакой PolyGloss~\cite{Tan2021CodeMixingOS}.
\parДля перевода слов на другие языки мы используем модель машинного перевода M2M 100 от компании Facebook~\cite{Fan2020BeyondEM}.
Она содержит 418 миллионов параметров.
\par Псевдокод функции ExtendSlotLabels можно найти в приложении (Алгоритм~\eqref{alg:algorithm3}).

\begin{algorithm}
    \caption{Word-level атака}
    \begin{algorithmic}
        \Require{Словарь переводов с исходного на встраиваемый язык $\mathbb{T}$}
        \Function{GetCandidates}{$\mathcal{M}$, x, y, token\_id}
            \ind\If{x[token\_id] in $\mathbb{T}[\mathbb{L}]$}
                    \\
                    \ind\ind tokens = $\mathbb{T}[\mathbb{L}]$[x[token\_id]]\\
                    \ind\ind x[token\_id] = tokens\\
                    \ind\ind y[token\_id] = ExtendSlotLabels(y[token\_id], len(tokens))
            \EndIf \\
            \ind\Return x, y
        \EndFunction
    \end{algorithmic}\label{alg:algorithm1}
\end{algorithm}

\begin{table}[H]
	\resizebox{\textwidth}{!}{
		\begin{tabular}{|>{\bfseries}l|ccccccccc|}
			\hline
			Utterance en &what & are & the & flights & from & tacoma & to & san & jose \\ \hline
			Utterance adv &what & are & El & vuelos & from & tacoma & to & san & jose \\ \hline
		\end{tabular}
	}\caption{Пример 1 атаки модели XLM-RoBERTa (xlm-r) word-level атакой.}\label{tab:table18}
\end{table}
\begin{table}[H]
	\resizebox{\textwidth}{!}{
		\begin{tabular}{|>{\bfseries}l|ccccccccccc|}
			\hline
			Utterance en &i & would & like & flight & information & from & phoenix & to & denver &   &   \\ \hline
			Utterance adv &y & would & como & flight & Información & from & El & Phoenix & para & El & Denver \\ \hline
		\end{tabular}
	}\caption{Пример 2 атаки модели XLM-RoBERTa (xlm-r) word-level атакой.}\label{tab:table19}
\end{table}
\begin{table}[H]
	\resizebox{\textwidth}{!}{
		\begin{tabular}{|>{\bfseries}l|ccccccccc|}
			\hline
			Utterance en &what & are & the & flights & from & las & vegas & to & ontario \\ \hline
			Utterance adv &what & sind & Die & flights & from & las & VEGAS & to & ontario \\ \hline
		\end{tabular}
	}\caption{Пример 3 атаки модели XLM-RoBERTa (xlm-r) word-level атакой.}\label{tab:table20}
\end{table}


\subsubsection{Phrase-level атака}
Второй предлагаемый нами вариант атаки заключается в генерации эквивалентов из других языков с помощью построения выравниваний между предложениями на разных языках.
Одно предложение является переводом другого, для перевода можно использовать ту же модель машинного перевода~\cite{Fan2020BeyondEM}, однако мы пользуемся тем, что у нас уже параллельный корпус.
Кандидаты для каждого токена определяются как токены из предложения на встраиваемом языке, в которые был выровнен токен.
Этот вариант атаки схож с атакой Bumblebee~\cite{Tan2021CodeMixingOS}.
\parДля построения выравниваний мы используем модель awesome-align на основе m-BERT~\cite{Dou2021WordAB}.

\begin{algorithm}
    \caption{Phrase-level атака}
    \begin{algorithmic}
        \Require{Выравнивание предложения на исходном языке к предложению на целевом языке $\mathbb{A}$}
        \Function{GetCandidates}{$\mathcal{M}$, x, y, token\_id}
            \If{x[token\_id] in $\mathbb{A}[\mathbb{L}]$}
                \\
                \ind\ind tokens = $\mathbb{A}[\mathbb{L}]$[x[token\_id]]\\
                \ind\ind x[token\_id] = tokens\\
                \ind\ind y[token\_id] = ExtendSlotLabels(y[token\_id], len(tokens))
            \EndIf \\
            \ind\Return x, y
        \EndFunction
    \end{algorithmic}\label{alg:algorithm2}
\end{algorithm}

\begin{table}[H]
	\resizebox{\textwidth}{!}{
		\begin{tabular}{|>{\bfseries}l|ccccccccc|}
			\hline
			Utterance en &please & find & flights & available & from & kansas & city & to & newark \\ \hline
			Utterance adv &encontre & find & flights & disponíveis & from & kansas & City & para & Newark \\ \hline
		\end{tabular}
	}\caption{Пример 1 атаки модели XLM-RoBERTa (xlm-r) phrase-level атакой.}\label{tab:table21}
\end{table}
\begin{table}[H]
	\resizebox{\textwidth}{!}{
		\begin{tabular}{|>{\bfseries}l|ccccccccc|}
			\hline
			Utterance en &what & are & the & flights & from & tacoma & to & san & jose \\ \hline
			Utterance adv &cuáles & are & the & flights & from & tacoma & a & san & jose \\ \hline
		\end{tabular}
	}\caption{Пример 2 атаки модели XLM-RoBERTa (xlm-r) phrase-level атакой.}\label{tab:table22}
\end{table}
\begin{table}[H]
	\resizebox{\textwidth}{!}{
		\begin{tabular}{|>{\bfseries}l|ccccccccc|}
			\hline
			Utterance en &show & flights & saturday & evening & from & st. & louis & to & burbank \\ \hline
			Utterance adv &show & flights & sábado & evening & from & St. & Louis & para & Burbank \\ \hline
		\end{tabular}
	}\caption{Пример 3 атаки модели XLM-RoBERTa (xlm-r) phrase-level атакой.}\label{tab:table23}
\end{table}


\subsection{Метод адверсариального предобучения для защиты от адверсариальных атак}
В своей работе мы предлагаем метод защиты от предложенных выше адверсариальных атак.
Гипотеза заключается в том, что данный метод позволит увеличить качество не только на адверсариальных пертурбациях, но и на реальных данных со смешением кодов.
\parПредлагаемый нами метод адверсариального предобучения состоит из нескольких шагов:
\begin{enumerate}
    \item Генерация выборки для задачи маскированного моделирования языка.
    \item Дообучение тела мультиязычной модели на сгенерированной выборке в режиме предсказания маскированных токенов.
    \item Загрузка дообученного тела модели перед началом обучения для задачи одновременного заполнения слотов и классификации интентов.
\end{enumerate}

\begin{algorithm}
    \caption{Генерация адверсариальной выборки}
    \begin{algorithmic}
        \Require{Обучающая выборка датасета $X$, набор встраиваемых языков $\mathbb{L}_1, \dots \mathbb{L}_n$}
        \Ensure{Адверсариальная выборка $X'$} \\
        X' = [~]
        \For{$\mathbb{L}$ in $\mathbb{L}_1, \dots \mathbb{L}_n$}
            \For{x in X}
                \For{i in permutation(len(x))}
                    \\
                    \ind\ind\ind Candidates = GetCandidates($\mathcal{M}$, x, y, token\_id = i)
                    \ind\If{Candidates and $\unif$(0, 1) > 0.5}
                            \\
                            \ind\ind\ind\ind x, \_ = random.choice(Candidates)
                    \EndIf
                \EndFor \\
                \ind\ind X`.append(x)
            \EndFor
        \EndFor \\
        \Return X'
    \end{algorithmic}\label{alg:algorithm4}
\end{algorithm}

\subsubsection{Генерация адверсариальной выборки}
Для генерации выборки используется адаптация алгоритма phrase-level адверсариальной атаки (Алгоритм~\eqref{alg:algorithm4}).
Разница заключается в том, что токены заменяются на их эквиваленты с некоторой вероятностью.
Таким образом, для генерации выборки не требуется обученная модель.
\parВыборка является конкатенацией сгенерированных выборок для всех шести языков кроме английского представленных в датасете.
Каждая из подвыборок генерируется встраиванием целевого языка в обучающую выборку датасета MultiAtis++ на английском языке.
Псевдокод функции GetCandidates представлен в секции про атаки (Алгоритм~\eqref{alg:algorithm2}).
\parПосле генерации у нас получается 6 подвыборок по 4884 предложения в каждой.
Итоговая выборка состоит из 29304 предложений, мы делим эту выборку в отношении 9 к 1 на обучающую и тестовую.

\subsubsection{Дообучение тела модели}
После генерации адверсариальной выборки мы дообучаем предобученную мультиязычную модель на этой выборке.
Модель обучается в режиме задачи маскированного моделирования языка.
\parДля обучения модели для такой задачи мы отбираем 15\% токенов и предсказываем их с помощью модели.
80\% отобранных токенов заменяются на токен маски, 10\% заменяются на случайные слова из словаря, остальные 10\% остаются неизменными.
Мы дообучаем обе мультиязычные модели m-BERT и XLM-RoBERTa с одинаковыми гиперпараметрами - 10 эпох с размером батча 64 и длиной шага $10^{-5}$.
После дообучения мы сохраняем тело модели для дальнейшего использования.

% TODO: дописать про лосс.

\subsubsection{Загрузка дообученного тела модели}
Перед обучением мультиязычной модели для задачи одновременного заполнения слотов и классификации интентов мы загружаем дообученное тело модели.

\subsection{Результаты}
% TODO: решить что кинуть в приложения, а что нет.

\subsubsection{Решение задачи классификации интентов и заполнения слотов}

% TODO: графики обучения моделек.
% TODO: мы решили задачу всё тип-топ (сравнить с моделькой, что училась только на английском).

\begin{table}[H]
	\resizebox{\textwidth}{!}{
		\begin{tabular}{|>{\bfseries}l|c|c|c|c|c|c|c|c|}
			\hline
			& en & de & es & fr & ja & pt & zh & avg \\ \hline
			xlm-r&$0.980$ & $0.975$ & $0.968$ & $0.972$ & $0.977$ & $0.970$ & $0.968$ & $0.973$ \\ \hline
			m-bert&$0.977$ & $0.977$ & $0.963$ & $0.966$ & $0.959$ & $0.967$ & $0.962$ & $0.967$ \\ \hline
			xlm-r en&$0.903$ & $0.885$ & $0.882$ & $0.879$ & $0.830$ & $0.846$ & $0.856$ & $0.869$ \\ \hline
			m-bert en&$0.951$ & $0.828$ & $0.865$ & $0.877$ & $0.750$ & $0.853$ & $0.795$ & $0.845$ \\ \hline
		\end{tabular}
	}\caption{Сравнение моделей между собой \textbf{на тестовой выборке} датасета MultiAtis++ по метрике \textbf{Intent accuracy}. По колонкам языки тестовых подвыборок, по рядам тестируемые модели.}\label{tab:table0}
\end{table}
\begin{table}[H]
	\resizebox{\textwidth}{!}{
		\begin{tabular}{|>{\bfseries}l|c|c|c|c|c|c|c|c|}
			\hline
			& en & de & es & fr & ja & pt & zh & avg \\ \hline
			xlm-r&$0.945$ & $0.937$ & $0.902$ & $0.924$ & $0.931$ & $0.927$ & $0.948$ & $0.931$ \\ \hline
			m-bert&$0.947$ & $0.951$ & $0.895$ & $0.931$ & $0.933$ & $0.924$ & $0.944$ & $0.932$ \\ \hline
			xlm-r en&$0.871$ & $0.702$ & $0.750$ & $0.629$ & $0.535$ & $0.715$ & $0.707$ & $0.701$ \\ \hline
			m-bert en&$0.902$ & $0.553$ & $0.787$ & $0.524$ & $0.643$ & $0.517$ & $0.678$ & $0.658$ \\ \hline
		\end{tabular}
	}\caption{Сравнение моделей между собой \textbf{на тестовой выборке} датасета MultiAtis++ по метрике \textbf{Slots F1 score}. По колонкам языки тестовых подвыборок, по рядам тестируемые модели.}\label{tab:table1}
\end{table}
\begin{table}[H]
	\resizebox{\textwidth}{!}{
		\begin{tabular}{|>{\bfseries}l|c|c|c|c|c|c|c|c|}
			\hline
			& en & de & es & fr & ja & pt & zh & avg \\ \hline
			xlm-r&$0.829$ & $0.824$ & $0.689$ & $0.804$ & $0.740$ & $0.813$ & $0.800$ & $0.786$ \\ \hline
			m-bert&$0.849$ & $0.866$ & $0.654$ & $0.811$ & $0.738$ & $0.801$ & $0.775$ & $0.785$ \\ \hline
			xlm-r en&$0.558$ & $0.310$ & $0.354$ & $0.167$ & $0.000$ & $0.321$ & $0.093$ & $0.257$ \\ \hline
			m-bert en&$0.675$ & $0.192$ & $0.415$ & $0.191$ & $0.095$ & $0.181$ & $0.195$ & $0.278$ \\ \hline
		\end{tabular}
	}\caption{Сравнение моделей между собой \textbf{на тестовой выборке} датасета MultiAtis++ по метрике \textbf{Semantic accuracy}. По колонкам языки тестовых подвыборок, по рядам тестируемые модели.}\label{tab:table2}
\end{table}


\subsubsection{Качество моделей после адверсариальных атак}

% TODO: мы проатаковали модельки, всё тип-топ (сравнить между собой, показать как качество упало).

\begin{table}[H]
	\resizebox{\textwidth}{!}{
		\begin{tabular}{|>{\bfseries}l|c|c|c|c|c|c|c|}
			\hline
			& de & es & fr & ja & pt & zh & avg \\ \hline
			xlm-r&$0.931$ & $0.877$ & $0.849$ & $0.825$ & $0.901$ & $0.872$ & $0.876$ \\ \hline
			m-bert&$0.893$ & $0.891$ & $0.872$ & $0.820$ & $0.853$ & $0.852$ & $0.863$ \\ \hline
			xlm-r en&$0.809$ & $0.783$ & $0.774$ & $0.677$ & $0.554$ & $0.728$ & $0.721$ \\ \hline
			m-bert en&$0.811$ & $0.760$ & $0.793$ & $0.723$ & $0.760$ & $0.777$ & $0.771$ \\ \hline
		\end{tabular}
	}\caption{Сравнение моделей между собой после word-level атаки на тестовую выборку датасета MultiAtis++ по метрике \textbf{Intent accuracy}. По колонкам встраиваемые языки, по рядам тестируемые модели.}\label{tab:table3}
\end{table}
\begin{table}[H]
	\resizebox{\textwidth}{!}{
		\begin{tabular}{|>{\bfseries}l|c|c|c|c|c|c|c|}
			\hline
			& de & es & fr & ja & pt & zh & avg \\ \hline
			xlm-r&$0.767$ & $0.589$ & $0.603$ & $0.552$ & $0.598$ & $0.747$ & $0.642$ \\ \hline
			m-bert&$0.685$ & $0.517$ & $0.510$ & $0.428$ & $0.494$ & $0.684$ & $0.553$ \\ \hline
			xlm-r en&$0.642$ & $0.467$ & $0.499$ & $0.508$ & $0.543$ & $0.641$ & $0.550$ \\ \hline
			m-bert en&$0.539$ & $0.385$ & $0.419$ & $0.362$ & $0.391$ & $0.585$ & $0.447$ \\ \hline
		\end{tabular}
	}\caption{Сравнение моделей между собой после word-level атаки на тестовую выборку датасета MultiAtis++ по метрике \textbf{Slots F1 score}. По колонкам встраиваемые языки, по рядам тестируемые модели.}\label{tab:table4}
\end{table}
\begin{table}[H]
	\resizebox{\textwidth}{!}{
		\begin{tabular}{|>{\bfseries}l|c|c|c|c|c|c|c|}
			\hline
			& de & es & fr & ja & pt & zh & avg \\ \hline
			xlm-r&$0.343$ & $0.109$ & $0.086$ & $0.170$ & $0.077$ & $0.278$ & $0.177$ \\ \hline
			m-bert&$0.228$ & $0.083$ & $0.058$ & $0.081$ & $0.038$ & $0.213$ & $0.117$ \\ \hline
			xlm-r en&$0.201$ & $0.057$ & $0.056$ & $0.013$ & $0.023$ & $0.042$ & $0.065$ \\ \hline
			m-bert en&$0.136$ & $0.029$ & $0.032$ & $0.005$ & $0.011$ & $0.115$ & $0.055$ \\ \hline
		\end{tabular}
	}\caption{Сравнение моделей между собой после word-level атаки на тестовую выборку датасета MultiAtis++ по метрике \textbf{Semantic accuracy}. По колонкам встраиваемые языки, по рядам тестируемые модели.}\label{tab:table5}
\end{table}

\begin{table}[H]
	\resizebox{\textwidth}{!}{
		\begin{tabular}{|>{\bfseries}l|c|c|c|c|c|c|c|}
			\hline
			& de & es & fr & ja & pt & zh & avg \\ \hline
			xlm-r&$0.954$ & $0.946$ & $0.928$ & $0.952$ & $0.964$ & $0.950$ & $0.949$ \\ \hline
			m-bert&$0.948$ & $0.935$ & $0.939$ & $0.951$ & $0.940$ & $0.934$ & $0.941$ \\ \hline
			xlm-r en&$0.808$ & $0.836$ & $0.740$ & $0.750$ & $0.442$ & $0.784$ & $0.727$ \\ \hline
			m-bert en&$0.809$ & $0.833$ & $0.834$ & $0.805$ & $0.861$ & $0.829$ & $0.829$ \\ \hline
		\end{tabular}
	}\caption{Сравнение моделей между собой после phrase-level атаки на тестовую выборку датасета MultiAtis++ по метрике \textbf{Intent accuracy}. По колонкам встраиваемые языки, по рядам тестируемые модели.}\label{tab:table6}
\end{table}
\begin{table}[H]
	\resizebox{\textwidth}{!}{
		\begin{tabular}{|>{\bfseries}l|c|c|c|c|c|c|c|}
			\hline
			& de & es & fr & ja & pt & zh & avg \\ \hline
			xlm-r&$0.802$ & $0.829$ & $0.751$ & $0.444$ & $0.813$ & $0.609$ & $0.708$ \\ \hline
			m-bert&$0.784$ & $0.804$ & $0.758$ & $0.450$ & $0.783$ & $0.619$ & $0.700$ \\ \hline
			xlm-r en&$0.627$ & $0.704$ & $0.569$ & $0.365$ & $0.680$ & $0.561$ & $0.584$ \\ \hline
			m-bert en&$0.539$ & $0.699$ & $0.531$ & $0.366$ & $0.530$ & $0.563$ & $0.538$ \\ \hline
		\end{tabular}
	}\caption{Сравнение моделей между собой после phrase-level атаки на тестовую выборку датасета MultiAtis++ по метрике \textbf{Slots F1 score}. По колонкам встраиваемые языки, по рядам тестируемые модели.}\label{tab:table7}
\end{table}
\begin{table}[H]
	\resizebox{\textwidth}{!}{
		\begin{tabular}{|>{\bfseries}l|c|c|c|c|c|c|c|}
			\hline
			& de & es & fr & ja & pt & zh & avg \\ \hline
			xlm-r&$0.511$ & $0.511$ & $0.336$ & $0.115$ & $0.522$ & $0.209$ & $0.368$ \\ \hline
			m-bert&$0.487$ & $0.438$ & $0.344$ & $0.114$ & $0.433$ & $0.256$ & $0.345$ \\ \hline
			xlm-r en&$0.163$ & $0.229$ & $0.099$ & $0.013$ & $0.070$ & $0.064$ & $0.106$ \\ \hline
			m-bert en&$0.122$ & $0.219$ & $0.085$ & $0.041$ & $0.087$ & $0.106$ & $0.110$ \\ \hline
		\end{tabular}
	}\caption{Сравнение моделей между собой после phrase-level атаки на тестовую выборку датасета MultiAtis++ по метрике \textbf{Semantic accuracy}. По колонкам встраиваемые языки, по рядам тестируемые модели.}\label{tab:table8}
\end{table}


\subsubsection{Влияние метода адверсариального предобучения}

% TODO: результаты обучения MLM моделей (вставить графики).
% TODO: графики обучения для моделек с защитой.
% TODO: мы обучили и защитили модельки, всё почти тип-топ (сравнить между собой, показать, что где-то качество выросло, где-то упало).

\begin{table}[H]
	\resizebox{\textwidth}{!}{
		\begin{tabular}{|>{\bfseries}l|c|c|c|c|c|c|c|c|}
			\hline
			& en & de & es & fr & ja & pt & zh & avg \\ \hline
			xlm-r adv&$0.981$ & $0.974$ & $0.964$ & $0.976$ & $0.972$ & $0.967$ & $0.967$ & $0.972$ \\ \hline
			m-bert adv&$0.975$ & $0.976$ & $0.964$ & $0.972$ & $0.960$ & $0.970$ & $0.962$ & $0.968$ \\ \hline
			xlm-r en + adv&$0.928$ & $0.890$ & $0.913$ & $0.872$ & $0.789$ & $0.881$ & $0.816$ & $0.870$ \\ \hline
			m-bert en + adv&$0.959$ & $0.848$ & $0.901$ & $0.893$ & $0.719$ & $0.901$ & $0.759$ & $0.854$ \\ \hline
		\end{tabular}
	}\caption{Сравнение моделей с защитой между собой на тестовой выборке датасета MultiAtis++ по метрике \textbf{Intent accuracy}. По колонкам языки тестовых подвыборок, по рядам тестируемые модели.}\label{tab:table9}
\end{table}
\begin{table}[H]
	\resizebox{\textwidth}{!}{
		\begin{tabular}{|>{\bfseries}l|c|c|c|c|c|c|c|c|}
			\hline
			& en & de & es & fr & ja & pt & zh & avg \\ \hline
			xlm-r adv&$0.947$ & $0.940$ & $0.906$ & $0.929$ & $0.928$ & $0.929$ & $0.946$ & $0.932$ \\ \hline
			m-bert adv&$0.950$ & $0.942$ & $0.900$ & $0.928$ & $0.935$ & $0.920$ & $0.946$ & $0.932$ \\ \hline
			xlm-r en + adv&$0.888$ & $0.729$ & $0.788$ & $0.623$ & $0.447$ & $0.743$ & $0.718$ & $0.705$ \\ \hline
			m-bert en + adv&$0.900$ & $0.566$ & $0.759$ & $0.557$ & $0.416$ & $0.554$ & $0.604$ & $0.622$ \\ \hline
		\end{tabular}
	}\caption{Сравнение моделей с защитой между собой на тестовой выборке датасета MultiAtis++ по метрике \textbf{Slots F1 score}. По колонкам языки тестовых подвыборок, по рядам тестируемые модели.}\label{tab:table10}
\end{table}
\begin{table}[H]
	\resizebox{\textwidth}{!}{
		\begin{tabular}{|>{\bfseries}l|c|c|c|c|c|c|c|c|}
			\hline
			& en & de & es & fr & ja & pt & zh & avg \\ \hline
			xlm-r adv&$0.833$ & $0.832$ & $0.693$ & $0.813$ & $0.746$ & $0.809$ & $0.792$ & $0.788$ \\ \hline
			m-bert adv&$0.861$ & $0.854$ & $0.682$ & $0.805$ & $0.738$ & $0.799$ & $0.797$ & $0.791$ \\ \hline
			xlm-r en + adv&$0.613$ & $0.397$ & $0.404$ & $0.109$ & $0.005$ & $0.419$ & $0.136$ & $0.298$ \\ \hline
			m-bert en + adv&$0.674$ & $0.266$ & $0.366$ & $0.265$ & $0.004$ & $0.278$ & $0.136$ & $0.284$ \\ \hline
		\end{tabular}
	}\caption{Сравнение моделей с защитой между собой на тестовой выборке датасета MultiAtis++ по метрике \textbf{Semantic accuracy}. По колонкам языки тестовых подвыборок, по рядам тестируемые модели.}\label{tab:table11}
\end{table}

\begin{table}[H]
	\resizebox{\textwidth}{!}{
		\begin{tabular}{|>{\bfseries}l|c|c|c|c|c|c|c|}
			\hline
			& de & es & fr & ja & pt & zh & avg \\ \hline
			xlm-r adv&$0.930$ & $0.907$ & $0.883$ & $0.833$ & $0.911$ & $0.869$ & $0.889$ \\ \hline
			m-bert adv&$0.919$ & $0.913$ & $0.883$ & $0.881$ & $0.902$ & $0.848$ & $0.891$ \\ \hline
			xlm-r en adv&$0.874$ & $0.813$ & $0.830$ & $0.793$ & $0.834$ & $0.796$ & $0.824$ \\ \hline
			m-bert en adv&$0.852$ & $0.824$ & $0.805$ & $0.710$ & $0.857$ & $0.779$ & $0.804$ \\ \hline
		\end{tabular}
	}\caption{Сравнение моделей \textbf{с защитой} между собой после \textbf{word-level} атаки на тестовую выборку датасета MultiAtis++ по метрике \textbf{Intent accuracy}. По колонкам встраиваемые языки, по рядам тестируемые модели.}\label{tab:table12}
\end{table}
\begin{table}[H]
	\resizebox{\textwidth}{!}{
		\begin{tabular}{|>{\bfseries}l|c|c|c|c|c|c|c|}
			\hline
			& de & es & fr & ja & pt & zh & avg \\ \hline
			xlm-r adv&$0.771$ & $0.598$ & $0.592$ & $0.543$ & $0.604$ & $0.731$ & $0.640$ \\ \hline
			m-bert adv&$0.687$ & $0.507$ & $0.533$ & $0.518$ & $0.557$ & $0.675$ & $0.580$ \\ \hline
			xlm-r en adv&$0.662$ & $0.491$ & $0.485$ & $0.516$ & $0.536$ & $0.668$ & $0.560$ \\ \hline
			m-bert en adv&$0.565$ & $0.407$ & $0.384$ & $0.493$ & $0.398$ & $0.582$ & $0.471$ \\ \hline
		\end{tabular}
	}\caption{Сравнение моделей \textbf{с защитой} между собой после \textbf{word-level} атаки на тестовую выборку датасета MultiAtis++ по метрике \textbf{Slots F1 score}. По колонкам встраиваемые языки, по рядам тестируемые модели.}\label{tab:table13}
\end{table}
\begin{table}[H]
	\resizebox{\textwidth}{!}{
		\begin{tabular}{|>{\bfseries}l|c|c|c|c|c|c|c|}
			\hline
			& de & es & fr & ja & pt & zh & avg \\ \hline
			xlm-r adv&$0.367$ & $0.136$ & $0.094$ & $0.119$ & $0.085$ & $0.278$ & $0.180$ \\ \hline
			m-bert adv&$0.297$ & $0.111$ & $0.072$ & $0.155$ & $0.068$ & $0.216$ & $0.153$ \\ \hline
			xlm-r en adv&$0.252$ & $0.078$ & $0.065$ & $0.007$ & $0.075$ & $0.143$ & $0.103$ \\ \hline
			m-bert en adv&$0.200$ & $0.075$ & $0.042$ & $0.054$ & $0.032$ & $0.122$ & $0.088$ \\ \hline
		\end{tabular}
	}\caption{Сравнение моделей \textbf{с защитой} между собой после \textbf{word-level} атаки на тестовую выборку датасета MultiAtis++ по метрике \textbf{Semantic accuracy}. По колонкам встраиваемые языки, по рядам тестируемые модели.}\label{tab:table14}
\end{table}

\begin{table}[H]
	\resizebox{\textwidth}{!}{
		\begin{tabular}{|>{\bfseries}l|c|c|c|c|c|c|c|}
			\hline
			& de & es & fr & ja & pt & zh & avg \\ \hline
			xlm-r adv&$0.951$ & $0.944$ & $0.927$ & $0.962$ & $0.958$ & $0.951$ & $0.949$ \\ \hline
			m-bert adv&$0.960$ & $0.956$ & $0.948$ & $0.951$ & $0.956$ & $0.954$ & $0.954$ \\ \hline
			xlm-r en adv&$0.873$ & $0.854$ & $0.878$ & $0.829$ & $0.865$ & $0.837$ & $0.856$ \\ \hline
			m-bert en adv&$0.838$ & $0.869$ & $0.846$ & $0.755$ & $0.906$ & $0.774$ & $0.831$ \\ \hline
		\end{tabular}
	}\caption{Сравнение моделей \textbf{с защитой} между собой после \textbf{phrase-level} атаки на тестовую выборку датасета MultiAtis++ по метрике \textbf{Intent accuracy}. По колонкам встраиваемые языки, по рядам тестируемые модели.}\label{tab:table15}
\end{table}
\begin{table}[H]
	\resizebox{\textwidth}{!}{
		\begin{tabular}{|>{\bfseries}l|c|c|c|c|c|c|c|}
			\hline
			& de & es & fr & ja & pt & zh & avg \\ \hline
			xlm-r adv&$0.808$ & $0.840$ & $0.762$ & $0.433$ & $0.817$ & $0.621$ & $0.713$ \\ \hline
			m-bert adv&$0.793$ & $0.844$ & $0.782$ & $0.458$ & $0.815$ & $0.631$ & $0.720$ \\ \hline
			xlm-r en adv&$0.648$ & $0.756$ & $0.610$ & $0.380$ & $0.681$ & $0.580$ & $0.609$ \\ \hline
			m-bert en adv&$0.615$ & $0.754$ & $0.606$ & $0.350$ & $0.618$ & $0.537$ & $0.580$ \\ \hline
		\end{tabular}
	}\caption{Сравнение моделей \textbf{с защитой} между собой после \textbf{phrase-level} атаки на тестовую выборку датасета MultiAtis++ по метрике \textbf{Slots F1 score}. По колонкам встраиваемые языки, по рядам тестируемые модели.}\label{tab:table16}
\end{table}
\begin{table}[H]
	\resizebox{\textwidth}{!}{
		\begin{tabular}{|>{\bfseries}l|c|c|c|c|c|c|c|}
			\hline
			& de & es & fr & ja & pt & zh & avg \\ \hline
			xlm-r adv&$0.539$ & $0.521$ & $0.370$ & $0.142$ & $0.543$ & $0.273$ & $0.398$ \\ \hline
			m-bert adv&$0.538$ & $0.540$ & $0.462$ & $0.139$ & $0.560$ & $0.274$ & $0.419$ \\ \hline
			xlm-r en adv&$0.268$ & $0.311$ & $0.177$ & $0.021$ & $0.249$ & $0.132$ & $0.193$ \\ \hline
			m-bert en adv&$0.252$ & $0.347$ & $0.164$ & $0.036$ & $0.275$ & $0.113$ & $0.198$ \\ \hline
		\end{tabular}
	}\caption{Сравнение моделей \textbf{с защитой} между собой после \textbf{phrase-level} атаки на тестовую выборку датасета MultiAtis++ по метрике \textbf{Semantic accuracy}. По колонкам встраиваемые языки, по рядам тестируемые модели.}\label{tab:table17}
\end{table}

