\subsection{Обучение моделей на датасете ATIS seven languages}
В своей работе мы обучаем языковые модели решать задачу одновременной классификации интентов и разметки слотов в предложении.
Эта задача заключается в определении желаемой цели запроса пользователя по предложению и классификации слов в предложении.

\subsubsection{Датасет}
В качестве датасета в своей работе мы выбрали датасет ATIS seven languages~\cite{Xu2020EndtoEndSA}.
В этом датасете представлены семь языков из трёх языковых семей —
Индо-Европейская (английский, немецкий, французский, испанский, португальский), Японо-рюкюская (японский) и Сино-тибетская (китайский).
Датасет является параллельным корпусом для задачи классификации интентов и разметки слотов - в 2020 году он был переведён с английского языка на остальные шесть.
В обучающей выборке содержится 4978 предложений для каждого языка, в тестовой 893 предложения для каждого языка.
\parКаждый объект в датасете состоит из предложения, меток слов и интента.
Перед началом работы с датасетом мы произвели предварительную очистку —
убрали из обучающей и тестовой выборок объекты, для которых на любом из семи языков количество слов и количество слотов не совпадали.
Таким образом, в обучающей выборке осталось 4884 объекта для каждого языка, в тестовой выборке 755 объектов для каждого языка.
Для составления списка используемых слотов и интентов использовалась обучающая выборка на английском языке.
Мы использовали 121 различную метку слотов и 23 различных метки интентов.
Список id используемых объектов, а также списки используемых слотов и интентов можно найти в приложении.

\subsubsection{Архитектура модели}
В своей работе мы решаем задачу одновременной классификации интентов и разметки слотов в предложении с помощью одной модели.
Модель имеет два выхода, первый предсказывает интенты, второй предсказывает метки слов.
В качестве рассматриваемых архитектур были выбраны модели m-BERT~\cite{devlin-etal-2019-bert} и XLM-RoBERTa~\cite{Conneau2020UnsupervisedCR}.
Обе эти модели являются одними из самых сильных мультиязычных моделей на текущий момент.
Каждая из них предобучена на более чем ста языках.
% TODO: дописать про размеры моделей.

\subsubsection{Обучение}
В своей работе мы будем сравнивать модели, обученные на всей обучающей выборке и только на части обучающей выборки на английском языке.
Таким образом мы сможем проверить гипотезу о наличии кросс-язычных знаний у моделей.
Тестовая выборка, которая будет нас интересовать в данном контексте состоит из всех семи языков, но мы оцениваем качество на каждом языке отдельно. \\
Каждая из моделей обучалась с одинаковыми гиперпараметрами - 10 эпох на обучающей выборке с длиной шага обучения $10^{-5}$ и размером батча в 64 объекта.
% TODO: что-нибудь еще написать + добавить графики обучения.
% TODO: сделать результаты на каждом языке в тесте.

\subsection{Адверсариальные атаки}
В своей работе мы предлагаем два варианта gray-box адверсариальных атак — во время выполнения атаки мы имеем доступ к ошибке модели.
Мы стремимся создать атаку такого рода, чтобы результирующая адверсариальная пертурбация предложения была как можно ближе к реалистичным предложениям со смешением кодов.
Для этого мы заменяем часть токенов в предложении на их эквиваленты из других языков.
Оценка качества на таких адверсариальных атаках может выступать в роли оценки снизу на качество соответствующих моделей в аналогичных задачах при наличии реального смешения кодов во входных данных. \\
Так как большинство людей, которые могут использовать смешение кодов в своей речи билингвы, то в основном смешение кодов происходит между парой языков~\cite{bilinguals}.
Таким образом, в своей работе мы предлагаем анализировать атаки состоящие во встраивании одного языка в другой.

\subsubsection{Общий вид атаки}
Общий принцип атаки одинаковый для обоих предлагаемых вариантов.
Разница между методами заключается в способе генерации кандидатов на замену токену на $i-$ой позиции.
В своей работе мы предлагаем следующий вид атаки — пусть мы имеем целевую модель, пару пример-метка и встраиваемый язык~\eqref{alg:algorithm}.
Тогда мы перебираем токены в предложении в случайном порядке и стремимся заменить токен на его эквивалент из встраиваемого языка.
Если это приведёт к увеличению ошибки модели, то мы заменяем токен на предложенного кандидата.

\begin{algorithm}
    \caption{Адверсариальная атака, общая схема}
    \begin{algorithmic}
        \Require{Пара пример-метка x, y; целевая модель $\mathcal{M}$; встраиваемый язык $\mathbb{L}$}
        \Ensure{Адверсариальный пример $x'$} \\
        $\mathcal{L}_{x}$ = GetLoss($\mathcal{M}$, x, y)
        \For{i in permutation(len(x))}
            \\
            \ind Candidates = GetCandidates($\mathcal{M}$, x, y, token\_id = i) \\
            \ind Losses = GetLoss($\mathcal{M}$, Candidates)
            \ind\If{Candidates and max(Losses) > $\mathcal{L}_{x}$}
                    \\
                    \ind\ind$\mathcal{L}_{x}$ = max(Losses) \\
                    \ind\ind x, y = Candidates[argmax(Losses)]
            \EndIf
        \EndFor \\
        \Return x
    \end{algorithmic}\label{alg:algorithm}
\end{algorithm}

\subsubsection{Word level атака}
Первый предлагаемый нами вариант атаки заключается в генерации эквивалентов из других языков с помощью перевода токенов на соответствующие языки.
Этот вариант является грубой оценкой снизу, так как он не учитывает контекста предложений и не учитывает многозначность слов. \\
Для перевода слов на другие языки мы используем модель машинного перевода M2M 100 от компании Facebook ~\cite{Fan2020BeyondEM}.
Она содержит 418 миллионов параметров.

\begin{algorithm}
    \caption{Word-level атака}
    \begin{algorithmic}
        \Require{Словарь переводов с исходного на встраиваемый язык $\mathbb{T}$}
        \Function{GetCandidates}{$\mathcal{M}$, x, y, token\_id}
            \ind\If{x[token\_id] in $\mathbb{T}[\mathbb{L}]$}
                    \\
                    \ind\ind token = $\mathbb{T}[\mathbb{L}]$[x[token\_id]]\\
                    \ind\ind x[token\_id] = token
            \EndIf \\
            \Return x, y
        \EndFunction
    \end{algorithmic}\label{alg:algorithm1}
\end{algorithm}

% TODO: пример атаки.

\subsubsection{Phrase-level атака}
Второй предлагаемый нами вариант атаки заключается в генерации эквивалентов из других языков с помощью построения выравниваний между предложениями на разных языков.
Кандидаты для каждого токена определяются как токены из предложения на встраиваемом языке, в которые был выровнен токен. \\
Для построения выравниваний мы используем модель awesome-align на основе m-BERT~\cite{Dou2021WordAB}.

\begin{algorithm}
    \caption{Word-level атака}
    \begin{algorithmic}
        \Require{Выравнивание предложения на исходном языке к предложению на целевом языке $\mathbb{A}$}
        \Function{GetCandidates}{$\mathcal{M}$, x, y, token\_id}
            \ind\If{x[token\_id] in $\mathbb{A}[\mathbb{L}]$}
                    \\
                    \ind\ind tokens = $\mathbb{A}[\mathbb{L}]$[x[token\_id]]\\
                    \ind\ind x[token\_id] = tokens\\
                    \ind\ind y[token\_id] = ExtendSlotLabels(y[token\_id], len(tokens))
            \EndIf \\
            \Return x, y
        \EndFunction
    \end{algorithmic}\label{alg:algorithm2}
\end{algorithm}

% TODO: пример атаки.

\subsection{Метод защиты от адверсариальных атак}
В своей работе мы предлагаем метод защиты от предложенных выше адверсариальных атак.
Гипотеза заключается в том, что данный метод позволит увеличить качество не только на адверсариальных пертурбациях, но и на реальных данных со смешением кодов.

\subsubsection{Метод адверсариального предобучения}

\subsection{Адверсариальные атаки на защищенные модели}
В данной секции мы сравним результаты

\subsubsection{Сравнение на тестовой выборке}

\begin{table}[H]
    \resizebox{\textwidth}{!}{
        \begin{tabular}{|>{\bfseries}l|c|c|c|c|}
            \hline
            & xlm-r   & xlm-r en & xlm-r adv & xlm-r en + adv \\ \hline
            Intent accuracy   & $0.980$ & $0.902$  & $0.980$   & $0.963$        \\ \hline
            Slot F1 score     & $0.944$ & $0.870$  & $0.948$   & $0.899$        \\ \hline
            Semantic accuracy & $0.826$ & $0.559$  & $0.842$   & $0.670$        \\ \hline
            Loss              & $0.317$ & $0.729$  & $0.293$   & $0.575$        \\ \hline
        \end{tabular}
    }\caption{Таблица сравнения моделей XLM-R между собой на тестовой выборке}\label{tab:table}
\end{table}

\begin{table}[H]
    \resizebox{\textwidth}{!}{
        \begin{tabular}{|>{\bfseries}l|c|c|c|c|}
            \hline
            & m-bert  & m-bert en & m-bert adv & m-bert en + adv \\ \hline
            Intent accuracy   & $0.979$ & $0.952$   & $0.975$    & $0.948$         \\ \hline
            Slot F1 score     & $0.947$ & $0.899$   & $0.952$    & $0.908$         \\ \hline
            Semantic accuracy & $0.854$ & $0.672$   & $0.846$    & $0.690$         \\ \hline
            Loss              & $0.353$ & $0.584$   & $0.328$    & $0.577$         \\ \hline
        \end{tabular}
    }\caption{Таблица сравнения моделей M-BERT между собой на тестовой выборке}\label{tab:table2}
\end{table}

% TODO: написать текст про сравнение и выводы из этого.

\subsubsection{Сравнение для word-level атаки}

\begin{table}[H]
    \resizebox{\textwidth}{!}{
        \begin{tabular}{|>{\bfseries}l|c|c|c|c|}
            \hline
            & xlm-r             & xlm-r en          & xlm-r adv         & xlm-r en + adv    \\ \hline
            Intent accuracy   & $0.885 \pm 0.035$ & $0.727 \pm 0.081$ & $0.893 \pm 0.037$ & $0.851 \pm 0.035$ \\ \hline
            Slot F1 score     & $0.642 \pm 0.080$ & $0.550 \pm 0.069$ & $0.651 \pm 0.078$ & $0.568 \pm 0.065$ \\ \hline
            Semantic accuracy & $0.179 \pm 0.097$ & $0.065 \pm 0.059$ & $0.191 \pm 0.105$ & $0.089 \pm 0.067$ \\ \hline
            Loss              & $2.627 \pm 0.727$ & $3.232 \pm 0.809$ & $2.424 \pm 0.667$ & $2.624 \pm 0.612$ \\ \hline
        \end{tabular}
    }\caption{Таблица сравнения моделей XLM-R после word-level атаки}\label{tab:table3}
\end{table}

\begin{table}[H]
    \resizebox{\textwidth}{!}{
        \begin{tabular}{|>{\bfseries}l|c|c|c|c|}
            \hline
            & m-bert            & m-bert en         & m-bert adv        & m-bert en + adv   \\ \hline
            Intent accuracy   & $0.866 \pm 0.028$ & $0.771 \pm 0.032$ & $0.863 \pm 0.023$ & $0.781 \pm 0.046$ \\ \hline
            Slot F1 score     & $0.556 \pm 0.095$ & $0.444 \pm 0.083$ & $0.585 \pm 0.086$ & $0.489 \pm 0.064$ \\ \hline
            Semantic accuracy & $0.120 \pm 0.079$ & $0.056 \pm 0.053$ & $0.145 \pm 0.088$ & $0.090 \pm 0.065$ \\ \hline
            Loss              & $3.137 \pm 0.701$ & $3.335 \pm 0.662$ & $2.878 \pm 0.611$ & $3.019 \pm 0.512$ \\ \hline
        \end{tabular}
    }\caption{Таблица сравнения моделей M-BERT после word-level атаки}\label{tab:table4}
\end{table}

% TODO: написать текст про сравнение и выводы из этого.

\subsubsection{Сравнение для phrase-level атаки}

\begin{table}[H]
    \resizebox{\textwidth}{!}{
        \begin{tabular}{|>{\bfseries}l|c|c|c|c|}
            \hline
            & xlm-r             & xlm-r en          & xlm-r adv         & xlm-r en + adv    \\ \hline
            Intent accuracy   & $0.947 \pm 0.006$ & $0.728 \pm 0.136$ & $0.954 \pm 0.009$ & $0.864 \pm 0.040$ \\ \hline
            Slot F1 score     & $0.708 \pm 0.140$ & $0.581 \pm 0.109$ & $0.721 \pm 0.148$ & $0.641 \pm 0.129$ \\ \hline
            Semantic accuracy & $0.366 \pm 0.156$ & $0.105 \pm 0.074$ & $0.405 \pm 0.164$ & $0.228 \pm 0.138$ \\ \hline
            Loss              & $2.026 \pm 1.152$ & $2.860 \pm 0.826$ & $1.992 \pm 1.248$ & $1.943 \pm 0.743$ \\ \hline
        \end{tabular}
    }\caption{Таблица сравнения моделей XLM-R после phrase-level атаки}\label{tab:table5}
\end{table}

\begin{table}[H]
    \resizebox{\textwidth}{!}{
        \begin{tabular}{|>{\bfseries}l|c|c|c|c|}
            \hline
            & m-bert            & m-bert en         & m-bert adv        & m-bert en + adv   \\ \hline
            Intent accuracy   & $0.942 \pm 0.004$ & $0.828 \pm 0.020$ & $0.950 \pm 0.005$ & $0.818 \pm 0.035$ \\ \hline
            Slot F1 score     & $0.700 \pm 0.127$ & $0.536 \pm 0.096$ & $0.728 \pm 0.137$ & $0.577 \pm 0.150$ \\ \hline
            Semantic accuracy & $0.348 \pm 0.127$ & $0.113 \pm 0.055$ & $0.406 \pm 0.158$ & $0.198 \pm 0.113$ \\ \hline
            Loss              & $2.118 \pm 1.143$ & $2.474 \pm 0.591$ & $1.935 \pm 1.135$ & $2.252 \pm 0.825$ \\ \hline
        \end{tabular}
    }\caption{Таблица сравнения моделей M-BERT после phrase-level атаки}\label{tab:table6}
\end{table}

% TODO: написать текст про сравнение и выводы из этого.

\subsection{Результаты}

% TODO: написать текст про результаты сравнения.

\subsubsection{Кросс-язычные знания в моделях}
\subsubsection{Качество моделей после адверсариальных атак}
\subsubsection{Влияние метода адверсариального предобучения}
