Последние несколько лет стали прорывными в области мультиязычных моделей и их обобщающей способности для других языков~\cite{Conneau2020UnsupervisedCR,devlin-etal-2019-bert,Liu2020WhatMM,Wu2019BetoBB}.
Огромные мультиязычные модели выучивают универсальные языковые представления, что помогает им демонстрировать удивительные способности к переносу знаний с одного языка на другой.
Простое дообучение предобученных моделей для какой-либо задачи на языке с большим количеством данных позволяет достичь хорошего качества на других языках.
\parОднако простой перенос между языками недостаточен для систем обработки естественного языка для понимания мультиязычных пользователей.
Во многих сообществах в мире достаточно часто явление смешения кодов.
Смешение кодов — это процесс, когда человек спонтанно смешивает различные языки внутри одного предложения или фразы.
Такой феномен может проявляться как в письменной, так и в устной речи.
Таким образом, важно сделать языковую модель устойчивой к смешению языков, чтобы модель адекватно работала со входными данными.
\parНесмотря на то, что реальные данные со смешением кодов очень важны для оценки качества языковых моделей, такие данные очень тяжело собирать и размечать в большом количестве.
\parВ своей работе мы предполагаем, что качество моделей на адверсариальных атаках может служить нижней оценкой на реальное качество модели.
Если языковая модель успешно справляется с адверсариальными пертурбациями со смешением кодов, то и в реальной жизни она будет успешно обрабатывать данные от мультиязычных пользователей.
\parВ своей работе мы:
\begin{itemize}
    \item Решаем задачу одновременного детектирования намерений пользователя и заполнения слотов для диалоговых помощников с помощью мультиязычных языковых моделей.
    \item Предлагаем две адверсариальные атаки по методу серого ящика — во время атаки мы имеем доступ к ошибке модели на заданных данных.
    Насколько нам известно, это одни из первых мультиязычных адверсариальных атак для вышеописанной задачи.
    \item Предлагаем метод адверсариального предобучения.
\end{itemize}
\parВ результате работы мы ожидаем получить следующие результаты:
\begin{itemize}
    \item Мультиязычные модели обучены решать задачу заполнения слотов и классификации интентов.
    \item Проведены две адверсариальные атаки на каждую модель и замерено качество моделей на адверсариальных данных.
    \item Оценено влияние метода адверсариального предобучения на качество моделей на тестовой выборке и после адверсариальных атак.
\end{itemize}
\parВсе свои эксперименты мы будем проводить с современными мультиязычными моделями - m-BERT~\cite{devlin-etal-2019-bert} и XLM-RoBERTa~\cite{Conneau2020UnsupervisedCR}.
В качестве датасета мы будем использовать корпус MultiAtis++~\cite{Xu2020EndtoEndSA}.
\parАктуальность темы подтверждается повышенным интересом со стороны научного сообщества.
После начала работы над исследованием вышло как минимум три статьи на эту тему — две в марте~\cite{Krishnan2021MultilingualCF,Tan2021CodeMixingOS} и одна в конце апреля~\cite{santy-etal-2021-bertologicomix} 2021 года.
