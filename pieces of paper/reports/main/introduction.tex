Последние несколько лет стали прорывными в области мультиязычных моделей и их обобщающей способности для других языков~\cite{Liu2020WhatMM,Wu2019BetoBB}.
Огромные мультиязычные модели выучивают универсальные языковые представления, что помогает им демонстрировать удивительные способности к переносу знаний с одного языка на другой.
Простое дообучение предобученных моделей для какой-либо задачи на языке с большим количеством данных позволяет достичь хорошего качества на других языках.
\parОднако простой перенос между языками недостаточен для систем обработки естественного языка для понимания мультиязычных пользователей.
Во многих сообществах в мире достаточно часто явление смешения кодов.
Смешение кодов — это процесс, когда человек спонтанно смешивает различные языки внутри одного предложения или фразы.
Такой феномен может проявляться как в письменной, так и в устной речи.
Таким образом, важно сделать языковую модель устойчивой к смешению языков, чтобы модель адекватно работала со входными данными.
\parНесмотря на то, что реальные данные со смешением языков очень важны для оценки способности языковых моделей работы со смешением кодов, такие данные очень тяжело собирать и размечать в большом количестве.
\parВ своей работе мы предполагаем, что качество моделей на адверсариальных атаках может служить нижней оценкой на реальное качество модели.
Если языковая модель успешно справляется с адверсариальными пертурбациями со смешением кодов, то и в реальной жизни она будет успешно обрабатывать данные от мультиязычных пользователей.
Таким образом, мы в своей работе:
\begin{itemize}
    \item Предлагаем две адверсариальные атаки по методу серого ящика — во время атаки мы имеем доступ к ошибке модели на заданных данных.
    Мы проводим атаки на мультиязычные модели для задачи одновременного детектирования намерений пользователя и заполнения слотов для диалоговых помощников, направленных на выполнение конкретной задачи.
    Насколько нам известно, это одни из первых мультиязычных адверсариальных атак для данной задачи.
    \item Предлагаем метод адверсариального предобучения и показываем, что он увеличивает качество моделей на наших атаках.
    \item Дополнительно исследуем перенос знаний для задачи одновременной классификации интентов и разметки слотов в предложении.
\end{itemize}
\parВсе свои эксперименты мы будем проводить с современными мультиязычными моделями - m-BERT~\cite{devlin-etal-2019-bert} и XLM-RoBERTa~\cite{Conneau2020UnsupervisedCR}.
В качестве датасета мы будем использовать корпус MultiAtis++~\cite{Xu2020EndtoEndSA}.
